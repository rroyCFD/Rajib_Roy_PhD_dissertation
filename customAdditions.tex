% This file contains all custom packages, commands, paths used in the dissertation%



%%rroy package addition
% used for subfloat option in figures
%\usepackage{subfig} % inculded in style file
% used for split command in equations
\usepackage{mathtools} 
% used in the textapprox command
\usepackage{textcomp} 
\usepackage{amssymb}
\usepackage{cancel} %% for \cancelto command
\usepackage{bm} %% \boldsymbol command
%\usepackage{float} %% for H figure placement specifier % inculded in style file
%\epstopdfsetup{update} % only regenerate pdf files when eps file is newer



%
%\newcommand{\name}[num][default]{definition}
\newcommand{\spaece}{\texttt{SPAeCE}\,\,}
\newcommand{\spaeceRC}{\texttt{SPAeCE-RC}\,\,}
\newcommand{\spaeceA}{\texttt{SPAeCE-A6}\,\,}
\newcommand{\spaeceARC}{\texttt{SPAeCE-A6RC}\,\,}
\newcommand{\spaeceAdivFree}{\texttt{SPAeCE-A6-divFree}\,\,}

\newcommand{\piso}{\texttt{PISO}\,\,}

\newcommand{\faceUnitNormal}{\mathbf{\hat{n}}}
\newcommand{\av}[1]{\overline{#1}}
\newcommand{\avb}[1]{\overline{\mathbf{#1}}}

\newcommand{\avtilde}[1]{\widetilde{\overline{#1}}}
\newcommand{\avprime}[1]{{\overline{#1}}^{\prime}}

\newcommand{\avbTilde}[1]{\widetilde{\avb{#1}}}
\newcommand{\avbPrime}[1]{{\overline{\mathbf{#1}}}^{\prime}}

\newcommand{\avU}[1]{\av{U}_{#1}}
\newcommand{\avu}[1]{\av{u}_{#1}}
\newcommand{\rst}[2]{\av{u_{#1}u_{#2}}}

\newcommand{\dd}[2]{\frac{\partial {#1}}{\partial {#2}}}

\newcommand{\ddt}[1] {\frac{\partial {#1}}{\partial t}}
\newcommand{\Ddt}[1] {\frac{\Delta {#1}}{\Delta t}}

\newcommand{\dx}[1]{\frac{\partial}{\partial x_{#1}}}
\newcommand{\ddx}[2]{\frac{\partial #1}{\partial x_{#2}}}
\newcommand{\ddxx}[3]{\frac{\partial^2 #1}{\partial x_{#2} \partial x_{#3}}}
\newcommand{\ddxj} {\frac{\partial}{\partial x_j}}

\newcommand{\eps}{\varepsilon}

\newcommand{\avT}[1]{\langle {#1} \rangle}
\newcommand{\RANSstress}[2]{\avT{{#1}^{\prime} {#2}^{\prime}}}

\newcommand{\half}{\frac{1}{2}}

\newcommand{\divergence}[1] { \nabla \cdot {#1}}
\newcommand{\grad}[1] {\nabla {#1}}
\newcommand{\curl}[1] {\nabla \times {#1}}
\newcommand{\laplacian}[1] {\Delta {#1}}
\newcommand{\nueffb}{\boldsymbol{\nu}_{eff}}

%\newcommand{\areaVec}{\mathbf{S}^f}
\newcommand{\areaVec}[1]{{S}_{#1}^f}
\newcommand{\magArea}{|\areaVec{}|}

\newcommand{\transDot}[1]{{#1}^T \cdot}
\newcommand{\shiftCS}{\, \mathbf{\Gamma}_{c \rightarrow f}}
\newcommand{\shiftSC}{\, \mathbf{\Gamma}_{f \rightarrow c}}


\newcommand{\textapprox}{\raisebox{0.5ex}{\texttildelow}}
\newcommand{\sumFV}{\sum_{f \textapprox faces(V_c)}}

%\newcommand{\inverse[1]}{#1\raisebox{1.15ex}{$\scriptscriptstyle-\!1$}}
\newcommand{\inverse[1]}{#1^{-1}}

\newcommand{\cubicRoot}[1]{\sqrt[\leftroot{-2}\uproot{2}{3}]{#1}}
%%%%%%%%%%%%%%%%%%%%%%%%%%%%%%%%%%%%

\newcommand{\imgHalfWidth}{0.475}
\newcommand{\imgOneThirdWidth}{0.31}
\newcommand{\imgFortySeven}{0.44462} % 46.8%
\newcommand{\imgFiftyThree}{0.50538} % 53.2%

%%%% From Bolund paper %%%%%

\newcommand{\Ua}[1]{\av{U}_{#1}}
\newcommand{\SGSstress}[2]{\avT{{#1}^{\prime} {#2}^{\prime}}}



\graphicspath{
{figures/}
{figures/BolundHill/}
}