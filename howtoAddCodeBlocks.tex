% This file is for chapter 2

\chapter{Theoretical Background}

\section{My First Section}

How about listings of computer programs? The main program
(\pc{main.c}) is very basic, as shown below. Note that unless your
advisor objects, program listings should be single-spaced, which
can be controlled with the \pc{\spacing} command as shown.  If you 
have longer and/or many program listings, it's usually better to 
place them in an appendix.

\begin{spacing}{1}
\begin{lstlisting}[caption={Main program for simple frame-based processing
  using ISRs.},label={cd:FrameMain_isr}]{}
#include "..\Common_Code\DSK_Config.h" 
#include "frames.h"

int main() {
    // initialize all buffers to 0
    ZeroBuffers();

    // initialize DSK for selected codec
    DSK_Init(CodecType, TimerDivider);

    // main loop here, process buffer when ready
    while(1) {
        if(IsBufferReady()) // process buffers in background
            ProcessBuffer();
    }
}
\end{lstlisting}
\end{spacing}
\noindent Wasn't that a nice program?  % see how to avoid an indented line?

How about some \ml\ code? Note you have to specify the language
since \ml\ wasn't the default language in the ``listings'' setup.

\begin{spacing}{1}
\begin{lstlisting}[language=matlab,%
caption={Simple \ml\ FIR filter example.}]{}
%  This m-file is used to convolve x[n] and B[n]
%
%  Assumes that both x[n] and B[n] start at n = 0
%
%  written by Dr. Thad B. Welch, PE {t.b.welch@ieee.org}
%  copyright 2001
%  completed on 13 December 2001 revision 1.0

% Simulation inputs
x = [1 2 3 0 1 -3 4 1];             % input vector x[n]
B = [0.25 0.25 0.25 0.25];          % FIR filter coefficients B[n]

% Calculated terms
PaddedX = [x zeros(1,length(B)-1)]; % zeros pads x[n] to flush the filter
n = 0:(length(x) + length(B) - 2);  % plotting index for the output
y = filter(B, 1, PaddedX);          % performs the convolution

% Simulation outputs
stem(n, y)                          % output plot generation
ylabel(`output values')
xlabel(`sample number')
\end{lstlisting}
\end{spacing}



\section{My Third Section}

Now let's see how a table is formatted. The minimum distance to a
nearest cluster point is given in Table~\ref{tb:results3}.
% uncomment the [!b] below if you *really want the table placed at the bottom of the page
\begin{table}%[!b]
\begin{center}
\caption[Results of the experiment testing for recognition of
occluded objects.]{Results of the third experiment, showing
Euclidean distance to nearest eigenspace model point. Smaller
numbers represent ``better'' recognition. This experiment tested
for recognition of occluded objects.\\}
 \label{tb:results3}
\begin{tabular}{c|c c c} \hline
  & Occluded F4 & Occluded F14 & Occluded Tornado \\ \hline
  Tornado & 13.8922 & 6.4154 & {\bf 68.9262}\\
  P51 & 6.7955 & 3.7622 & 53.9320 \\
  F4 & {\bf 5.7648} & 5.5956 & 48.3343 \\
  F14 & 6.9371 & {\bf 3.9662} & 48.2957 \\
  F22 & 4.8605 & 5.6179 & 45.3576 \\ \hline
\end{tabular}
\end{center}
\end{table}
Note that for a table environment, the caption comes \emph{before} the
definition of the table itself.

% Cheat to bring in other references
\nocite{*} % delete or comment this out.
