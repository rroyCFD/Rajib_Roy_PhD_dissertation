%%%%%%%%%%%%%%%%%%%%%%%%%%%%%%%%%%%%%%%%%%%%%%%%%%%%%%%%%%%%%%%%%%%%%%%%%%%%%
%% This file can be used to generate a list of symbols, abbreviations, etc.
%% as an appendix.
%%   For use with theses or dissertations using uwyo_thesis.sty
%%
%%   Version: 1.00
%%   Last modified: 1 May 2013
%%
%%%%%%%%%%%%%%%%%%%%%%%%%%%%%%%%%%%%%%%%%%%%%%%%%%%%%%%%%%%%%%%%%%%%%%%%%%%%%


%%%%%%%%%%%%%%%%%%%%%%%%
\chapter{Abbreviations, Acronyms, and Symbols}
\label{Abbreviations}  % for you to be able to refer to this appendix in the main text



% % % % % % % % % % % % % % % % % % % % % % % % % % % % %
{ % begin special environment for abbreviations  -- modify only with great care!

 \setlength{\parindent}{0pt}

%%%%% commands just for this file %%%%%%%%%%%%%%%%%%%%%%%
 \newcommand{\abbrltr}[1]{% command for letter section
   \bigskip
   \framebox{\textbf{#1}}
   \medskip
 }
 \newlength{\abbrwidth}
 \newlength{\abbrdef}
 \setlength{\abbrwidth}{0.6in}  % adjust for longest abbreviation
 \setlength{\abbrdef}{\textwidth}
 \addtolength{\abbrdef}{-\abbrwidth}
 \newcommand{\abbr}[2]{% command for an entry
   \begin{tabular}{p{\abbrwidth}p{\abbrdef}}
     \textbf{#1} & {#2}
   \end{tabular}

 }

%%%%%%%%%%%%%%%%%%%%%%%%%%%%%%%%%%%%%%%%%%%%%%%%%%%%%%%%%
%%%%%%%%%%%%%%%%%%%%%%%%
%% start appendix text %%
%%%%%%%%%%%%%%%%%%%%%%%%

This is a partial list of abbreviations, acronyms, and symbols used in the
text, provided in the hope that it will be helpful to some readers.

\abbrltr{Symbols}

\abbr{$(\:)$}{used for a continuous function.}

\abbr{$[\:\:]$}{used for a discrete function.}

\abbrltr{Greek Letters}

\abbr{$\alpha$}{feedback coefficient for simple IIR filters, such as those used for a
type of echo generation for guitar special effects.}

\abbr{$\lambda$}{wavelength.}

\abbr{$\pi$}{ratio of a circle circumference to diameter,
3.1415926535897932\ldots}

\abbr{$\tau$}{time constant.}

\abbr{$\omega$}{radian frequency.}

\abbrltr{A}

\abbr{$a$}{filter coefficient associated with an output term, $y$.
When used in a transfer function, the $a$ coefficients are
associated with the denominator of the transfer function.}

\abbr{$A$}{vector or array containing all of the $a$ terms.}


\abbr{ADC}{analog-to-digital converter.}

\abbr{AIC}{analog interface circuit (see codec).}

\abbr{AGC}{automatic gain control.}

\abbr{AM}{amplitude modulation.}

\abbr{ARM}{Advanced RISC Machine, a 32-bit reduced instruction set computer (RISC)
instruction set architecture (ISA) developed by ARM Holdings.}

\abbr{AWGN}{additive white Gaussian noise.}


\abbrltr{B}

\abbr{$b$}{filter coefficient associated with an input term, $x$.
When used in a transfer function, the $b$ coefficients are
associated with the numerator of the transfer function.}

\abbr{$B$}{vector or array containing all of the $b$ terms.}

\abbr{$BW$}{bandwidth of a bandpass signal.}

\abbr{BP}{bandpass.}

\abbr{BPF}{bandpass filter.}

\abbr{BPSK}{binary phase shift keying.}

\abbrltr{C}

\abbr{C}{value of capacitance.}

\abbr{CD-ROM}{Compact disk read-only memory.}

\abbr{CISC}{complex instruction set computer.}

\abbr{codec}{coder-decoder.  An integrated circuit that contains
both an ADC and a DAC.}

\abbr{CPU}{central processing unit.}

\abbrltr{D}

\abbr{DAC}{digital-to-analog converter.}

\abbr{D.C.}{direct current (0 Hz).}

\abbr{DDS}{direct digital synthesizer or direct digital
synthesis.}

\abbr{DF-I}{direct form I.}

\abbr{DF-II}{direct form II.}

\abbr{DFT}{discrete Fourier transform.}

\abbr{DMA}{direct memory access.}

\abbr{DSK}{DSP starter kit.}

\abbr{DSP}{digital signal processing or digital signal processor.}

\abbr{DTFT}{discrete-time Fourier transform.}

\abbr{DTMF}{dual-tone, multiple-frequency signals as defined by telephone companies.}

\abbrltr{E}

\abbr{EDMA}{enhanced direct memory access.}

\abbrltr{F}

\abbr{FCC}{Federal Communications Commission.}

\abbr{FIR}{finite impulse response.}

\abbr{FFT}{fast Fourier transform.}

\abbr{FT}{Fourier transform.}

\abbr{$\mathcal{F}$}{Fourier transform.}

\abbr{$\mathcal{F}^{-1}$}{inverse Fourier transform.}

\abbr{$f_h$}{highest or maximum frequency that is present in a
signal.}

\abbr{$F_s$}{sample frequency (samples/second) = $1/T_s$.}

\abbrltr{G}

\abbr{GPP}{general purpose processor.}

\abbr{GPU}{graphics processing unit.}

\abbrltr{H}

\abbr{$H(e^{j\omega})$}{discrete-time frequency response.}

\abbr{$H(j\omega)$}{continuous-time frequency response.}

\abbr{$h[n]$}{discrete-time impulse response or unit sample
response.}

\abbr{$h[t]$}{continuous-time impulse response.}

\abbr{$H(s)$}{continuous-time transfer or system function.}

\abbr{$H(z)$}{discrete-time transfer or system function.}

\abbr{HDTV}{high-definition television.}

\abbr{HP}{highpass.}

\abbr{HPF}{highpass filter.}

\abbr{HPI}{host port interface.}

\abbr{Hz}{hertz (cycles per second).}

\abbrltr{I}

\abbr{IF}{intermediate frequency.}

\abbr{IFFT}{inverse fast Fourier transform.}

\abbr{IIR}{infinite impulse response.}

\abbr{ISA}{instruction set architecture.}

\abbr{ISR}{interrupt service routine.}

\abbrltr{J}

\abbr{$j$}{$\sqrt{-1}$; identifies the imaginary part of a complex number. Some authors
use $i$ instead of $j$.}

\abbr{JTAG}{Joint Test Action Group, commonly used as the name of a debugging interface
for printed circuit boards and IC chips. Formalized as IEEE Std 1149.1 in 1990.}

%\abbrltr{K}

%\abbr{$k$}{dummy index of summation used in the convolution sum.}

\abbrltr{L}

\abbr{$\mathcal{L}$}{Laplace transform.}

\abbr{$\mathcal{L}^{-1}$}{inverse Laplace transform.}

\abbr{L}{value of inductance.}

\abbr{LFSR}{linear feedback shift register.}

\abbr{LP}{lowpass.}

\abbr{LPF}{lowpass filter.}

\abbr{LSB}{lower sideband, also used for least significant bit.}


\abbrltr{M}

\abbr{M}{the number of bands in a graphic equalizer.}

\abbr{MA}{moving average.}

\abbr{McASP}{multi-channel audio serial port.}

\abbr{McBSP}{multi-channel buffer serial port.}

\abbr{ML}{maximum likelihood.}

\abbrltr{N}

\abbr{$n$}{index or sample number.}

\abbr{$N$}{often used as filter order; in other contexts, it is used for the length of a
sequence, or for the length of an FFT.}

\abbr{NCO}{numerically controlled oscillator.}


\abbrltr{O}

\abbr{OMAP}{Open Multimedia Application Platform, a family of proprietary multi-core
system on chips (SoCs) by Texas Instruments.}


\abbrltr{P}

\abbr{PC}{personal computer.}

\abbr{PCM}{pulse code modulation.}

\abbr{PLL}{phase-locked loop.}

\abbr{PN}{pseudonoise.}

\abbr{PSK}{phase shift keying.}


\abbrltr{Q}

\abbr{$Q$}{quality factor.  $Q$ = bandwidth of a BP filter divided
by its center frequency.  The higher the value of $Q$, the more
selective the BP filter is.}

\abbr{QAM}{quadrature amplitude modulation.}

\abbr{QPSK}{quadrature phase shift keying.}

\abbrltr{R}

\abbr{$r$}{magnitude of a pole.  This is a measure of how far the
pole is from the origin.}

\abbr{R}{value of resistance.}

\abbr{RC}{resistor-capacitor.}

\abbr{RISC}{reduced instruction set computer.}

\abbr{RF}{radio frequency.}

\abbrltr{S}

\abbr{$s$}{the Laplace transform independent variable, $s = \sigma
+ j\omega$.}

\abbr{SoC}{system on chip.}

\abbrltr{T}

\abbr{$\tau$}{a dummy variable often used in convolution.}

\abbr{$t$}{time.}

\abbr{$T$}{period of a signal or function.}

\abbr{TED}{timing error detector.}

\abbr{$T_s$}{sample period = $1/F_s$.}

\abbr{TI}{Texas Instruments.}

\abbrltr{U}

\abbr{$u[n]$}{discrete-time unit step function.}

\abbr{$u(t)$}{unit step function.}

\abbr{U.S.}{United States (of America).}

\abbr{USB}{upper sideband; also used for Universal Serial Bus.}

\abbrltr{V}

\abbr{$V$}{voltage in Volts.}

\abbr{$V_{in}$}{input voltage.}

\abbr{$V_{out}$}{output voltage.}

\abbr{VLIW}{very long instruction word; this is a type of
architecture for DSPs.}

\abbrltr{W}

\abbr{winDSK}{original Windows-based program for the C31 DSK,
created by Mike Morrow.}

\abbr{winDSK6}{Windows-based program, the follow-on to winDSK, for the C6x DSK series. It
was created by Mike Morrow.}

\abbr{winDSK8}{Windows-based program, the follow-on to winDSK6, for the OMAP-L138
multi-core board). It was created by Mike Morrow.}

\abbrltr{X}

\abbr{$X(j\omega)$}{result of the Fourier transform
$\mathcal{F}\{x(t)\}$; it shows the frequency content of $x(t)$.}

\abbr{$x[n]$}{a discrete-time input signal.}

\abbr{$x(t)$}{a continuous-time input signal.}

%\newpage

\abbrltr{Y}

\abbr{$Y(j\omega)$}{result of the Fourier transform
$\mathcal{F}\{y(t)\}$; it shows the frequency content of $y(t)$.}

\abbr{$y[n]$}{a discrete-time output signal.}

\abbr{$y(t)$}{a continuous-time output signal.}

\abbrltr{Z}

\abbr{$z$}{the independent transform variable for discrete-time signals and systems.}

\abbr{$z^{-1}$}{a delay of 1 sample.}

\abbr{$Z_c$}{impedance of a capacitor.}

\abbr{$\mathcal{Z}$}{$z$-transform.}

\abbr{$\mathcal{Z}^{-1}$}{inverse $z$-transform.}


% % % % % % % % % % % % % % % % % % % % % % % % %
} % end environment for zero \parindent  -- do not remove this curly brace

%\mycleardoublepage

%%%%%%%%%%%%%%%%%%%%%%%%
%%% end appendix text %%%
%%%%%%%%%%%%%%%%%%%%%%%%


